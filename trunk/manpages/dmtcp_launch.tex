\begin{document}

\begin{Name}{1}{dmtcp\_launch}{Kapil Arya}{Distributed MultiThreaded CheckPointing}{dmtcp\_launch -- start a process under DMTCP control}
  \myProg{dmtcp_launch} -- start a process under DMTCP control.
\end{Name}

\section{Synopsis}
%%%%%%%%%%%%%%%%%%

\myProg{dmtcp_launch} \oArg{OPTIONS} \Arg{<command>} \oArg{args...}

\section{Description}
%%%%%%%%%%%%%%%%%%

\myProg{dmtcp_command} is a tool to send user commands to the
\myProg{dmtcp_coordinator} remotely.

\section{Options}
%%%%%%%%%%%%%%%%%%

\subsubsection{Connecting to the DMTCP Coordinator}
\begin{Description}
  \item[\Opt{-h}, \Opt{--host} \Arg{hostname} (environment variable DMTCP_HOST)]
    Hostname where dmtcp_coordinator is run (default: localhost)
  \item[\Opt{-p}, \Opt{--port} \Arg{port} (environment variable DMTCP_PORT)]
    Port where dmtcp_coordinator is run (default: 7779)
  \item[\Opt{-j}, \Opt{--join}]
    Join an existing coordinator, raise error if one doesn't
    already exist
  \item[\Opt{--new-coordinator}]
    Create a new coordinator at the given port. Fail if one
    already exists on the given port. The port can be specified
    with \Opt{--port}, or with environment variable DMTCP_PORT.  If no
    port is specified, start coordinator at a random port (same
    as specifying port '0').
  \item[\Opt{--no-coordinator}]
    Execute the process in stand-alone coordinator-less mode.
    Use dmtcp_command or \Opt{--interval} to request checkpoints.
  \item[\Opt{-i}, \Opt{-interval} \Arg{seconds} (environment variable DMTCP_CHECKPOINT_INTERVAL)]
    Time in seconds between automatic checkpoints.
    0 implies never (manual ckpt only); if not set and no env var,
    use default value set in dmtcp_coordinator or dmtcp_command.
    Not allowed if \Opt{--join} is specified
\end{Description}

\subsubsection{Checkpoint image generation}
\begin{Description}
  \item[\Opt{--gzip}, \Opt{--no-gzip} (environment variable DMTCP_GZIP=\Lbr01\Rbr)]
    Enable/disable compression of checkpoint images (default: 1 (enabled))\\
    WARNING:  gzip adds seconds.  Without gzip, ckpt is often $<$ 1s
  \item[\OptSArg{--ckptdir}{path} (environment variable DMTCP_CHECKPOINT_DIR)]
    Directory to store checkpoint images (default: curr dir at launch)
  \item[\Opt{--checkpoint-open-files}]
    Checkpoint open files and restore old working dir. (default: do neither)
  \item[\OptSArg{--mtcp-checkpoint-signal}{signum}]
      Signal number used internally by MTCP for checkpointing (default: 12)
\end{Description}

\subsubsection{Enable/disable plugins}
\begin{Description}
  \item[\OptSArg{--with-plugin}{plugins} (environment variable DMTCP_PLUGIN)]
    Colon-separated list of DMTCP plugins to be preloaded with DMTCP.
    (Absolute pathnames are required.)
  \item[\Opt{--batch-queue}, \Opt{--rm}]
    Enable support for resource managers (Torque PBS and SLURM).
    (default: disabled)
  \item[\Opt{--ptrace}]
    Enable support for PTRACE system call for gdb/strace etc. (default: disabled)
  \item[\Opt{--modify-env}]
    Update environment variables based on the environment on the restart host
    (e.g., DISPLAY=\$DISPLAY).  This can be set in a file dmtcp_env.txt.
    (default: disabled)
  \item[\Opt{--ib}, \Opt{--infiniband}]
    Enable InfiniBand plugin. (default: disabled)
  \item[\Opt{--disable-alloc-plugin} (environment variable DMTCP_ALLOC_PLUGIN=\Lbr01\Rbr)]
    Disable alloc plugin (default: enabled).
  \item[\Opt{--disable-dl-plugin} (environment variable DMTCP_DL_PLUGIN=\Lbr01\Rbr)]
    Disable dl plugin (default: enabled).
  \item[\Opt{--disable-all-plugins} (EXPERTS ONLY, FOR DEBUGGING)]
    Disable all plugins.
\end{Description}

\subsubsection{Other options}
\begin{Description}
  \item[\OptSArg{--prefix}{path}]
    Prefix where DMTCP is installed on remote nodes.
  \item[\OptSArg{--tmpdir}{path} (environment variable DMTCP_TMPDIR)]
    Directory to store temporary files
    (default: \$TMDPIR/dmtcp-\$USER@\$HOST or /tmp/dmtcp-\$USER@\$HOST)
  \item[\Opt{-q}, \Opt{--quiet} (or set environment variable DMTCP_QUIET = 0, 1, or 2)]
    Skip banner and NOTE messages; if given twice, also skip WARNINGs
  \item[\Opt{--help}] Print this message and exit.
  \item[\Opt{--version}] Print version information and exit.
\end{Description}
