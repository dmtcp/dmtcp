% DMTCP-Internal plugins:  make sure restart/resum operates after event:
%   RESTART/RESUME/REFILL/REGISTER\_NAME\_SERVICE\_DATA/SEND\_QUERIES
% DMTCP hierarchy:  DMTCP\_ROOT/include/dmtcp/plugin.h  (etc.) ; not dmtpplugin.h
% Do we install distro packages to use /usr/include/dmtcp/plugin.h for compatibility?
% Check documentation of events; make sure they correspond to implementations
% Section Using plugins to virtualize ids  (based on Torque?  Add to it, or skip.)
% Move DMTCP\_ROOT?dmtcp/doc to:  DMTCP\_ROOT?doc
% For KVM plugin, change MTCP break issue to be warning (not error)
% KVM plugin:  QEMU complains about negative value in signalfd

% TODO:  --with-icc
% TODO:  --vnc  (for dmtcp_checkpoint)
% TODO:  test suite from Joshua Louie (dlopen prior to main; epoll taking too long)
%
% UPDATE 'svn co' on dmtcp.sf.net;  Add latest DMTCP papers to dmtcp.sf.net
% ADD documents (design of DMTCP, design of plugins)

\documentclass{article}
\usepackage{fullpage}

\title{Tutorial for DMTCP Plugins}
\author{}
\date{March, 2013}

\begin{document}

\maketitle
\tableofcontents

\section{Introduction}

{\bf This is a reminder that there are more boldface below here. \\
  Also, see the comments in the LaTeX file for smaller issues for the 2.0 release. \\
  Also, must then create plugin-tutorial.pdf, and add to svn
}

Plugins enable one to modify the behavior of DMTCP.  Two of the most
common uses of plugins are:
\begin{enumerate}
\item to execute an additional action at the time of checkpoint, resume,
	or restart.
\item to add a wrapper function around a call to a library function (including
	wrappers around system calls).
\end{enumerate}

Plugins are used for a variety of purposes. The {\tt DMTCP\_ROOT/contrib} directory
contains packages that users have contributed to be used as part of DMTCP.
Several of these packaages are based on DMTCP plugins.  These include:
\begin{enumerate}
  \item a plugin to adapt DMTCP to automatically checkpoint and restart in
	conjunction with the TORQUE batch queue system.
  \item {\bf ADD WHEN COMPLETE:} a plugin to adapt DMTCP to automatically
	checkpoint and restart in conjunction with the Condor system
	for high throughput computing.
  \item {\bf ADD WHEN COMPLETE:} a plugin that allows DMTCP to checkpoint QEMU/KVM.
  \item {\bf ADD WHEN COMPLETE:} record-replay plugin  (later)
  \item {\bf ADD WHEN COMPLETE:} a package to adapt I/O to different
	environments, when a DMTCP checkpoint image is restarted with a
	modified filesystem, on a new host, etc.
  \item {\bf ADD WHEN COMPLETE:} a plugin that allows a Python program
	to call a Python function to checkpoint itself.
  \item {\bf ADD WHEN COMPLETE:} a plugin that enables one to checkpoint
	over the network to a remote file.
\end{enumerate}

Plugin code is expressive, while requiring only a modest number of lines
of code.  The plugins in the contrib directory vary in size from
{\bf FILL IN} to 1500 lines of code to implement a plugin for the
Torque batch queue.

Beginning with DMTCP version~2.0, much of DMTCP itself is also now
a plugin.  In this new design, the core DMTCP code is responsible
primarily for copying all of user space memory to a checkpoint
image file.  The remaining functions of DMTCP are handled by plugins,
found in {\tt DMTCP\_ROOT/plugin}.  Each plugin abstracts the essentials of a
different subsystem of the operating system and modifies its behavior
to accommodate checkpoint and restart.  Some of the subsystems for
which plugins have been written are: virtualization of process
and thread ids; files(open/close/dup/fopen/fclose/mmap/pty);
events (eventfd/epoll/poll/inotify/signalfd);
System~V IPC constructs (shmget/semget/msgget); TCP/IP
sockets (socket/connect/bind/listen/accept); and timers
(timer\_create/clock\_gettime).  (The indicated system calls are examples
only and not all-inclusive.)

\section{Anatomy of a plugin}

There are three primary mechanisms by which a plugin can modify the
behavior of either DMTCP or a target application.
\begin{description}
\item[Wrapper functions:]  One declares a wrapper function with the same
	name as an existing library function (including system calls in
	the run-time library).  The wrapper function can execute some
	prolog code, pass control to the ``real'' function,
	and then execute some epilog code.  Several plugins can wrap
	the same function in a nested manner.  One can also omit
	passing control to the ``real'' function, in order to shadow
	that function with an alternate behavior.
\item[Events:]  It is frequently useful to execute additional code
	at the time of checkpoint, or resume, or restart.  Plugins
	provide hook functions	to be called during these three events
	and numerous other important events in the life of a process.
\item[Coordinated checkpoint of distributed processes:]  DMTCP transparently
	checkpoints distributed computations across many nodes.
	At the time of checkpoint or restart, it may be necessary to
	coordinate information among the distributed processes.  For example,
	at restart time, an internal plugin of DMTCP allows the newly
	re-created processes to ``talk'' to their peers to discover the
	new network addresses of their peers.  This is important since
	a distributed computation may be restarted on a different cluster
	than its original one.
\item[Virtualization of ids:]  Ids (process id, {\bf FILL IN}) are
        assigned by the kernel, by a peer process, and by remote processes.
	Upon restart, the external agent may wish to assign a different
	id than the one assigned prior to checkpoint.  Techniques for
	virtualization of ids are described in Section~\ref{sec:virtualization}.
\end{description}

\section{Writing Plugins}

\subsection{Invoking a plugin}

Plugins are just dynamic run-time libraries (.so files).  They are
invoked at the beginning of a DMTCP computation as command-line options:
\hfill\break
\medskip\noindent
  \hspace{0.3truein} {\tt dmtcp\_checkpoint --with-plugin PLUGIN1.so:PLUGIN2.so myapp}
\medskip

Note that one can invoke multiple plugins as a colon-separated list.
One should either specify a full path for each plugin (each .so~library),
or else to define LD\_LIBRARY\_PATH to include your own plugin directory.

\subsection{The plugin mechanisms}

The mechanisms of plugins are most easily described through examples.
This tutorial will rely on the examples ins in {\tt DMTCP\_ROOT/test/plugin}.
To get a feeling for the plugins, one can ``cd'' into each of the
subdirectories and execute: ``{\tt make check}''.

\subsubsection{Plugin events}

For context, please scan the code of {\tt DMTCP\_ROOT/plugin/example/example.c}.
Executing ``{\tt make check}'' will demonstrate the intended behavior.
Plugin events are handled by including the function {\tt dmtcp\_process\_event}.
When a DMTCP plugin event occurs, DMTCP will call the
function {\tt dmtcp\_process\_event} for each plugin.
This function is required only if the plugin will handle plugin events.
See Appendix~A for further details.

{\tt
\begin{verbatim}
void dmtcp_process_event(DmtcpEvent_t event, DmtcpEventData_t *data)
{
  switch (event) {
  case DMTCP_EVENT_WRITE_CKPT:
    printf("\n*** The plugin is being called before checkpointing. ***\n");
    break;
  case DMTCP_EVENT_RESUME:
    printf("*** The plugin has now been checkpointed. ***\n");
    break;
  case DMTCP_EVENT_THREADS_RESUME:
    if (data->resumeInfo.isRestart) {
      printf("The plugin is now resuming or restarting from checkpointing.\n");
    } else {
      printf("The process is now resuming after checkpoint.\n");
    }
    break;
  ...
  default:
    break;
  }
  NEXT_DMTCP_PROCESS_EVENT(event, data);
}
\end{verbatim}
}

{\bf ACTUALLY, I THINK MY
  EXAMPLE plugin was calling the wrong event name.  I'll look at this
  more carefully later. -- Gene}

{\bf I'LL FINISH WRITING LATER.}


Plugin events:

*** The model for events:  When a DMTCP event occurs, DMTCP calls ROUTINE
in each plugin, in the order that the plugins were loaded, offering each plugin
a chance to handle the event.  If ROUTINE is not defined in a plugin, DMTCP
skips calling that plugin.  When ROUTINE is called, it is given the unique
event id, and a switch statement can decide whether to take any special action.
If no action is taken, ROUTINE returns XXX, and the next plugin is offered
a chance to handle the event.  If a plugin does handle the event, a typical
user code fragment will:
A.  optionally carry out any pre-processing steps
B.  optionally ask DMTCP to invoke the next event handler
C.  optionall carry out any post-processing steps

If all three steps are invoked, this effectively creates a wrapper function around
any later plugins that handle the same event.  If step B is omitted,
no further plugins will be offered the opportunity to handle the event.

\subsubsection{Plugin wrapper functions}

In its simplest form, a wrapper function can be written as follows:

{\tt
\begin{verbatim}
unsigned int sleep(unsigned int seconds) {
  static unsigned int (*next_fnc)() = NULL; /* Same type signature as sleep */
  struct timeval oldtv, tv;
  gettimeofday(&oldtv, NULL);
  time_t secs = val.tv_sec;
  printf("sleep1: "); print_time(); printf(" ... ");
  unsigned int result = NEXT_FNC(sleep)(seconds);
  gettimeofday(&tv, NULL);
  printf("Time elapsed:  %f\n",
          (1e6*(val.tv_sec-oldval.tv_sec) + 1.0*(val.tv_usec-oldval.tv_usec)) / 1e6);
  print_time(); printf("\n");

  return result;
}
\end{verbatim}
}

In the above example, we could also shadow the standard ``sleep'' function
by our own implementation, if we omit the call to ``{NEXT\_FNC}''.

\noindent
To see a related example, try:
\hfill\break
\medskip\noindent
  \hspace{0.3truein} {\tt cd DMTCP\_ROOT/test/plugin/sleep1; make check}
\medskip

\noindent
Wrapper functions from distinct plugins can also be nested.  To see a nesting
of plugin sleep2 around sleep1, do:
\hfill\break
\medskip\noindent
  \hspace{0.3truein} {\tt cd DMTCP\_ROOT/test/plugin; make; cd sleep2; make check}
\medskip


Plugin wrappers:

Use sleep1/sleep2 for the example.

(see paper for other; mention tech. report ??, if ONWARD allows i.t)

\subsubsection{Plugin coordination among multiple or distributed processes}
\label{sec:publishSubscribe}

It is often the case that an external agent will assign a particular
initial id to your process, but later assign a different id on restart.
Each process must re-discover its peers at restart time, without knowing
the pre-checkpoint ids.

DMTCP provides a ``Publish/Subscribe'' feature to enable communication among
peer processes. Two plugin events allow user
plugins to discover peers and pass information among peers.
The two events are:  {\tt DMTCP\_EVENT\_REGISTER\_NAME\_SERVICE\_DATA}
 {\tt DMTCP\_EVENT\_SEND\_QUERIES}.  DMTCP guarantees to provide a global
barrier between the two events.

An example of how to use the Publish/Subscribe feature is contained
in the directory, DMTCP\_ROOT/test/plugin/example-db~.  The explanation below is best
understood in conjunction with reading that example.

A plugin processing {\tt DMTCP\_EVENT\_REGISTER\_NAME\_SERVICE\_DATA} should invoke: \\
int dmtcp\_send\_key\_val\_pair\_to\_coordinator(const void *key,
                                                   size\_t key\_len, \
                                                   const void *val, \
                                                   size\_t val\_len).

A plugin processing {\tt DMTCP\_EVENT\_SEND\_QUERIES} should invoke: \\
int dmtcp\_send\_query\_to\_coordinator(const void *key, size\_t key\_len,
                                            void *val, size\_t *val\_len).

\subsubsection{Using plugins to virtualize ids}

In this section, we consider a further complication.
If the user code or run-time library has cached that initial id, then
this presents a problem on restart.  Rather than create an independent
mechanism, this section shows how to handle this problem using existing tools.

{\bf Kapil, you said that Torque had a good example of this.  What is it?}

It is often the case that an external agent will assign a particular
initial id to your process, but later assign a different id on restart.
If the user code or run-time library has cached that initial id, then
this presents a problem on restart.  Each process must re-discover its
peers at restart time, without knowing the pre-checkpoint ids.

The solution is to virtualize the~id.
This mechanism is used internally in DMTCP to virtualize the
many ids provided by the kernel, by network host ids, and so on.
This section describes how your own plugin can take advantage of
the same mechanism.

A good example of the use of virtualization occurs in the Torque
plugin at {\tt DMTCP\_ROOT/contrib/torque}.  {\bf IS THIS TRUE?  WHAT IS
BEING VIRTUALIZED?}

\section{Writing Plugins that Virtualize IDs and Other Names}
\label{sec:virtualization}

Most writers of plugins can skip this section.  {\bf Virtualization
of names is required if ....}

\section{Caveats}

CAVEATS:
Does your plugin break normal DMTCP?  to test this, modify DMTCP, and copy your
plugin into {\tt DMTCP\_ROOT/lib}, and then run 'make check' for DMTCP as usual.

SHARED MEMORY REGIONS:
If two or more processes share a memory region, then the plugin writer must
be clear on whether DMTCP or the plugin has responsibility for restoring
the shared memory region.  Currently, EXPLAIN ....

Virtualizing long-lived objects:  HOWTO

INTERACTION OF MULTIPLE PLUGINS:
    For simple plugins, this issue can be ignored.  But if your plugin has
talbes with long-lived data, other wrappers may create additional instantiattions.
It is reasonable for them to do this for temporary data structures at
the time of checkpoint or at the time of restart.  But normally, such an
object, created when the checkpoint event occurs, should be destroyed before
creating the actual checkpoint image.  Similarly, at restart time, if new
instances are created, they should be destroyed before returning control
to the user threads.
    It is polite for a plugin to check the above restrictions.  If it is violated,
the plugin should print a warning about this.  This will help others, who
accidentally create long-lived objects at checkpoint- or restart-time,
without intending to.  If the other plugin intends this unusual behavior, one
can add a whitelist faeature for other plugings to declare such intentions.
    This small effort will provide a well-defined protocol that limits the
interaction between distinct plugins.  Your effort helps others to debug their
plugins, and a similar effort on their part will help you to debug your own plugin.

Putting a printf inside a plugin at the time of checkpoint is dangerous.  This
is because printf indirectly invokes a lock to prevent two threads from
printing simultaneously.  This causes the checkpoint thread to call a printf.
{\bf See README in test/plugin.}

At checkpoint time, the DMTCP user thread will stop on that same lock.  This
causes the two threads to deadliock.

This use of conflicting locks is a bug in DMTCP as of DMTCP-2.0.  It will
be fixed in the future version of DMTCP.

\appendix
\section{Appendix:  Plugin Manual}
\subsection{Plugin events}
% \addcontentsline{toc}{chapter}{Appendix A:  Plugin Events}

\subsubsection{dmtcp\_process\_event}

In order to handle DMTCP plugin events, a plugin must
define {\tt dmtcp\_process\_event}.

\begin{verbatim}
NAME
       dmtcp_process_event - Handle plugin events for this plugin

SYNOPSIS
       #include "dmtcp/plugin.h"

       void dmtcp_process_event(DmtcpEvent_t event, DmtcpEventData_t *data)

DESCRIPTION
       When a plugin event occurs, DMTCP will look for the symbol dmtcp_process_event
       in each plugin library.  If the symbol is found, that function will be called
       for the given plugin library.  DMTCP guarantees only to invoke the first such
       plugin library found in library search order.  Occurences of
       dmtcp_process_event in later plugin libraries will be called only if
       each previous function had invoked NEXT_DMTCP_PROCESS_EVENT.
       The argument, <event>, will be bound to the event being declared by
       DMTCP.  The argument, <data>, is required only for certain events.
       See the following section, ``Plugin Events'' for a list of all
       events.

SEE ALSO
       NEXT_DMTCP_PROCESS_EVENT
\end{verbatim}

\subsubsection{NEXT\_DMTCP\_PROCESS\_EVENT}

A typical definition of {\tt dmtcp\_process\_event} will invoke
{\tt NEXT\_DMTCP\_PROCESS\_EVENT}.

\begin{verbatim}
NAME
       NEXT_DMTCP_PROCESS_EVENT - call dmtcp_process_event in next plugin library

SYNOPSIS
       #include "dmtcp/plugin.h"

       void NEXT_DMTCP_PROCESS_EVENT(event, data)

DESCRIPTION
       This function must be invoked from within a plugin function library
       called dmtcp_process_event.  The arguments <event> and <data> should
       normally be the same arguments passed to dmtcp_process_event.

       NEXT_DMTCP_PROCESS_EVENT may be called zero or one times.  If invoked zero
       times, no further plugin libraries will be called to handle events.
       The behavior is undefined  if NEXT_DMTCP_PROCESS_EVENT is invoked more than
       once.  The typical usage of this function is to create a wrapper around
       the handling of the same event by later plugins.

SEE ALSO
       dmtcp_process_event
\end{verbatim}

If user-installed package, compile with {\tt -IDMTCP\_ROOT/dmtcp/include}~.

\subsubsection{Event Names}

The rest of this section defines plugin events.
The complete list of plugin events is always contained in
{\tt DMTCP\_ROOT/dmtcp/include/dmtcp/plugin.h}~.

DMTCP guarantees to call the dmtcp\_process\_event function of the plugin
when the specified event occurs.
If an event is thread-specific ({\bf
GIVE EXAMPLES}), DMTCP guarantees to call dmtcp\_process\_event within
the same thread.

{\bf DO I HAVE ALL THE THREAD-SPECIFIC EVENTS?}

Plugins that pass significant data through the data parameter
are marked with an asterisk: {}$^*$.
Most plugin events do not pass data through the data parameter.
Currently, the plugins
that use the data parameter {\bf use it to test if this is restart or resume??.
In this case, why don't we have a single function that every plugin can call
to test if this is during a restart or resume?}

Note that the events \\
   RESTART / RESUME / REFILL / REGISTER\_NAME\_SERVICE\_DATA / SEND\_QUERIES \\
should all be processed after the call to NEXT\_DMTCP\_PROCESS\_EVENT() in
order to guarantee that the internal DMTCP plugins have restored full
functionality.

\itemsep0pt
\subsubsection*{Checkpoint-Restart}
\begin{list}{}{\leftmargin=3em \itemindent=-2em}
\item
  DMTCP\_EVENT\_WRITE\_CKPT --- Invoked at final barrier before writing checkpoint
\item
  DMTCP\_EVENT\_RESTART --- Invoked at first barrier during restart of new process
\item
  DMTCP\_EVENT\_RESUME --- Invoked at first barrier during resume following checkpoint
\end{list}

\subsubsection*{Coordination of Multiple or Distributed Processes during Restart
	(see Appendix A.2.~Publish/Subscribe)}
\begin{list}{}{\leftmargin=3em \itemindent=-2em}
\item
  DMTCP\_EVENT\_REGISTER\_NAME\_SERVICE\_DATA$^*$ restart/resume
\item
  DMTCP\_EVENT\_SEND\_QUERIES$^*$ restart/resume
\end{list}

\subsubsection*{WARNING:  EXPERTS ONLY FOR REMAINING EVENTS}
\subsubsection*{Init/Fork/Exec/Exit}
\begin{list}{}{\leftmargin=3em \itemindent=-2em}
\item
  DMTCP\_EVENT\_INIT --- Invoked before main (in both the original program
and any new program called via exec)
\item
  DMTCP\_EVENT\_EXIT --- Invoked on call to exit/\_exit/\_Exit {\bf return from main?};
\item
  DMTCP\_EVENT\_PRE\_EXEC --- Invoked prior to call to exec
\item
  DMTCP\_EVENT\_POST\_EXEC --- Invoked before DMTCP\_EVENT\_INIT in new program
\item
  DMTCP\_EVENT\_ATFORK\_PREPARE --- Invoked before fork (see POSIX pthread\_atfork)
(was DMTCP\_EVENT\_PRE\_FORK)
\item
  DMTCP\_EVENT\_ATFORK\_PARENT --- Invoked after fork by parent (see POSIX
    pthread\_atfork)
\item
  DMTCP\_EVENT\_ATFORK\_CHILD --- Invoked after fork by child (see POSIX
    pthread\_atfork) \\
   (was DMTCP\_EVENT\_RESET\_ON\_FORK)
\end{list}

\subsubsection*{Barriers (finer-grainded control during checkpoint-restart)}
\begin{list}{}{\leftmargin=3em \itemindent=-2em}
\item
  DMTCP\_EVENT\_WAIT\_FOR\_SUSPEND\_MSG --- Invoked at barrier during
coordinated checkpoint
\item
  DMTCP\_EVENT\_SUSPENDED --- Invoked at barrier during coordinated checkpoint
\item
  DMTCP\_EVENT\_LEADER\_ELECTION --- Invoked at barrier during coordinated checkpoint
\item
  DMTCP\_EVENT\_DRAIN --- Invoked at barrier during coordinated checkpoint
\item
  DMTCP\_EVENT\_REFILL --- Invoked at first barrier during resume/restart of new process
\end{list}

\subsubsection*{Threads}
\begin{list}{}{\leftmargin=3em \itemindent=-2em}
\item
  DMTCP\_EVENT\_THREADS\_SUSPEND --- Invoked within checkpoint thread
	when all user threads have been suspended
\item
  DMTCP\_EVENT\_THREADS\_RESUME --- Invoked within checkpoint thread before
	any user threads are resumed. \\
	For debugging, consider calling the following code for this
        event:  {\tt static int x = 1; while(x);}
\item
  {\bf should we have separate DMTCP\_EVENT\_THREADS\_RESUME?  --- I vote yes.}
\item
  DMTCP\_EVENT\_PRE\_SUSPEND\_USER\_THREAD --- Each user thread invokes this prior
	to being suspended for a checkpoint
\item
  DMTCP\_EVENT\_RESUME\_USER\_THREAD --- Each user thread invokes this immediately
	after a resume or restart ({\tt isRestart()} available to plugin)
\item
  {\bf should we have separate DMTCP\_EVENT\_RESTART\_USER\_THREAD?  --- I vote yes.}
\item
  DMTCP\_EVENT\_THREAD\_START --- Invoked before start function given by clone
\item
  DMTCP\_EVENT\_THREAD\_CREATED --- Invoked within parent thread when clone call returns  (like parent for fork)
\item
  DMTCP\_EVENT\_PTHREAD\_START --- Invoked before start function given by pthread\_created
\item
  DMTCP\_EVENT\_PTHREAD\_EXIT --- Invoked before call to pthread\_exit
\item
  DMTCP\_EVENT\_PTHREAD\_RETURN --- Invoked in child thread when thread start function of pthread\_create returns
\item
\end{list}

\subsection{Publish/Subscribe}
% \addcontentsline({toc}{chapter}{Appendix B:  Publish/Subscribe}

Section~ref{sec:publishSubscribe} provides an explanation of the Publish/Subscribe
feature for coordination among peer processes at resume- or restart-time.
An example of how to use the Publish/Subscribe feature is contained
in the directory, test/plugin/example-db~.

The primary events and functions used in this feature are:

\noindent
{\tt DMTCP\_EVENT\_REGISTER\_NAME\_SERVICE\_DATA} \
int dmtcp\_send\_key\_val\_pair\_to\_coordinator(const void *key,
                                                   size\_t key\_len, \
                                                   const void *val, \
                                                   size\_t val\_len) \\
{\tt DMTCP\_EVENT\_SEND\_QUERIES} \\
int dmtcp\_send\_query\_to\_coordinator(const void *key, size\_t key\_len,
                                            void *val, size\_t *val\_len)

\subsection{Wrapper functions}
% \addcontentsline{toc}{chapter}{Appendix C:  Wrapper Functions}

{\bf fILL IN}

\subsection{Miscellaneous utility functions}

Numerous DMTCP utility functions are provided that can be called from within
dmtcp\_process\_event().  For a complete list, see
{\tt DMTCP\_ROOT/dmtcp/include/dmtcp/plugin.h}~.
The utility functions are still under active development, and may change
in small ways.

\end{document}
